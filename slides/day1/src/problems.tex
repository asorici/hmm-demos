\section{Theory of HMMs}
\label{sec:theory}

\subsection{The 3 things you want from an HMM}
\label{sec:problems}

\begin{frame}
  
  The 3 fundamental problems \parencite{rabiner1989tutorial}
  \begin{itemize}
  	\item Particularization of temporal inference problems to the HMM case
  	\item The restricted structure of the HMM allows for elegant implementations of all the basic algorithms
  \end{itemize}
  \pause

  \begin{block}{Evaluation Problem}
    Given a model and a sequence of observations, how do we compute the probability that the \alert{observed 
    sequence} was produced by the model?
  \end{block}
  \pause
  \begin{block}{Best Explanation of Observations Problem}
    Given a model and a sequence of observations how do we choose a corresponding sequence of \alert{states} which 
    \emph{gives meaning} to the observations? How do we \emph{uncover} the hidden part of the model?
  \end{block}
  \pause
  \begin{block}{Model Estimation (Training) Problem}
    Given some observed sequences, how do we adjust the \alert{parameters} of an HMM model that best tries to explain 
    the observations? 
    
  \end{block}

\end{frame}


\subsection{Mathematical Foundations for HMMs}
\label{sec:math}

\begin{frame}
  \frametitle{Elements of an HMM}
  \begin{itemize}
  \item $N$ Hidden States : $S_1,S_2, \ldots S_N$
  \item $M$ Observable Variables : $O_1, O_2, \ldots O_M$
  \end{itemize}
  Parameters:
  \begin{itemize}
  \item Transition Function / Matrix between states
  \item Emission probabilities
  \item Initial state probabilities
  \end{itemize}
\end{frame}
\begin{frame}
  \frametitle{Formalisation of the estimation problem}
  \begin{itemize}
  \item
  \end{itemize}
\end{frame}

\begin{frame}
  \frametitle{Formalisation of problem \# 2}
  \begin{itemize}
  \item $P(Q_1) = \displaystyle\sum_{x=\lbrace 1,2 \rbrace}^{N}P(Q_2) \theta\Pi$
  \item $P(Q_i \vert q_i = s_x) = i \times x \cdot i \ldots $
  \end{itemize}
\end{frame}

\begin{frame}
  \frametitle{Formalisation of parameters estimation problem}
  \begin{itemize}
  \item
  \end{itemize}
\end{frame}

\subsection{Notation Conventions \& Framework Description}
\label{sec:octave}

\begin{frame}
  \frametitle{Notation Conventions}

  
\end{frame}


\begin{frame}
  \frametitle{Variables in Octave}

  
\end{frame}
