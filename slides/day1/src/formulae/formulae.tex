\documentclass[10pt]{article}
\usepackage{amsmath}
\usepackage[utf8x]{inputenc}


\title{Formule de folosit in slide-uri}

\begin{document}

\maketitle

\section{De început}
\label{sec:start}


\begin{itemize}
\item Markov assumption
  \begin{equation}
    \label{eq:markovassump}
    \begin{split}
      P(q_t = s_j \vert q_{t-1}=s_i, q_{t-2}=s_k, \ldots ) = P(q_t =
      S_j \vert q_{t-1} = s_i) \\ \quad 1 \le i,j \le N
    \end{split}
  \end{equation}
\item Time independence
  \begin{equation}
    \label{eq:timeindep}
    a_{i,j}=P(q_t = s_j \vert q_{t-1} = s_i)
  \end{equation}
  \begin{equation}
    \label{eq:aispositive}
    a_{i,j} \ge 0
  \end{equation}
  \begin{equation}
    \label{eq:sumaij}
    \displaystyle\sum_{j=1}^{N}a_{i,j}=1, \quad 1 \le i \le N
  \end{equation}
\end{itemize}

\section{Parametrii modelului}
\label{sec:params}

\begin{itemize}
\item structure
  \begin{equation}
    \label{eq:lambda}
    \lambda = (A, B, \Pi)
  \end{equation}
\item pi
  \begin{equation}
    \label{eq:pi}
    \Pi = \lbrace \pi_i \rbrace,\quad 1 \le i \le N
  \end{equation}
  \begin{equation}
    \label{eq:pii}
    \pi_i = P(q_1 = s_i)
  \end{equation}
\item $A = \lbrace a_{i,j} \rbrace, \quad 1 \le i, j \le N$
  \begin{equation}
    \label{eq:transition}
    a_{i,j}=P(q_{t+1}=s_j \vert q_t = s_i),\quad 1 \le i, j \le N
  \end{equation}
\item $B = \lbrace b_{j,k} \rbrace, \substack{\quad 1 \le j \le N \\ 1
    \le l \le M}$
  \begin{equation}
    \label{eq:transition}
    b_{j,k}=b_{j}(v_k)=P(o_t = v_k \vert q_t = s_j),\quad 1 \le j \le N, 1 \le k, \le M
  \end{equation}
\item observațiile $O$
  \begin{equation}
    \label{eq:obs}
    O = [ o_1 o_2 \cdots o_T ]
  \end{equation}
\item variabilele de stare $Q$
  \begin{equation}
    \label{eq:states}
    Q = [ q_1 q_2 \cdots q_T ]
  \end{equation}
\item stările $s_1, s_2, \ldots, s_N$

\end{itemize}

\section{Niște probabilități}
\label{sec:probs}

\begin{itemize}
\item Probabilitatea unei secvențe observate, dat fiind $\lambda$
  \begin{equation}
    \label{eq:probobs}
    P(O \vert Q, \lambda)= \displaystyle\prod_{t=1}^{T}P(o_t \vert q_t, \lambda)
  \end{equation}

\end{itemize}

\section{Forward-Backward}
\label{sec:fb}

\subsection{Forward variables}
\label{sec:forward}
\begin{itemize}
\item alpha definition
  \begin{equation}
    \label{eq:alpha}
    \alpha_{t,i}=P(o_1,o_2,\ldots,o_t, q_t = S_i \vert \lambda)
  \end{equation}
\item alpha initialization
  \begin{equation}
    \label{eq:alpha_init}
    \alpha_{1,i}=\pi_ib_i(o_1), \quad 1 \le i \le N
  \end{equation}
\item alpha induction
  \begin{equation}
    \label{eq:alpha_induct}
    \alpha_{t+1,j}=\Big[ \displaystyle\sum_{i=1}^{N}\alpha_{t,j}a_{i,j}\Big] b_{j}(o_{t+1}), \quad \substack{1 \le t \le T-1, \\ 1 \le j \le N}
  \end{equation}
\item alpha termination
  \begin{equation}
    \label{eq:alpha_term}
    P(O|\lambda) = \displaystyle\sum_{i=1}^{N}\alpha_{T,i}
  \end{equation}
\end{itemize}
\subsection{Backward variables}
\label{sec:back}

\begin{itemize}
\item beta (backward variable)
  \begin{equation}
    \label{eq:beta}
    \beta_{t,i}=P(o_{t+1} o_{t+2} \cdots o_{T} \vert q_t = S_i, \lambda)
  \end{equation}
\item beta initialization
  \begin{equation}
    \label{eq:beta_init}
    \beta_{T,i}=1,\quad 1 \le i \le N
  \end{equation}
\item beta induction
  \begin{equation}
    \label{eq:beta_ind}
    \beta_{t,i}=\displaystyle\sum_{j=1}^{N}a_{i,j}b_j(o_{t+1})\beta_{t+1,j},
    \quad t = T-1, T-2, \ldots , 1, 1 \le i \le N
  \end{equation}
\end{itemize}

\section{Viterbi}
\label{sec:viterbi}

\begin{itemize}
\item Gamma
  \begin{equation}
    \label{eq:gamma_def}
    \gamma_{t,i}=P(q_t = s_i \vert O, \lambda)
  \end{equation}
  \begin{equation}
    \label{eq:gamma_formula}
    \gamma_{t,i}=\frac{\alpha_{t,i}\beta_{t,i}}{P(O\vert \lambda)} =
    \frac{\alpha_{t,i}\beta_{t,i}}{\displaystyle\sum_{k=1}^{N}\alpha_{t,k}\beta_{t,k}}
  \end{equation}
\item delta
  \begin{equation}
    \label{eq:delta}
    \delta_{t+1,j}=[\displaystyle \max_{i} \delta_{t,i} \cdot a_{ij}] \cdot b_j(o_{t+1})
  \end{equation}

\end{itemize}

\section{Bau-Bau-Baum-Welch}
\label{sec:baum}

\begin{itemize}
\item ksi definition
  \begin{equation}
    \label{eq:ksi_def}
    \xi_{t,i,j} = P(q_t=s_i,q_{t+1}=s_j \vert O, \lambda)
  \end{equation}
\item ksi computation
  \begin{equation}
    \label{eq:ksi_comp}
    \begin{split}
      \xi+{t,i,j} & = \frac{\alpha_{t,i}\cdot a_{i,j} \cdot
        b_j(o_{t+1}) \cdot \beta_{t+1,j}}
      {P(O \vert \lambda)} \\
      & = \frac{\alpha_{t,i}\cdot a_{i,j} \cdot b_j(o_{t+1}) \cdot
        \beta_{t+1,j}}{
        \displaystyle\sum_{k=1}^{N}\displaystyle\sum_{l=1}^{N}
        \alpha_{t,k}\cdot a_{k,l} \cdot b_l(o_{t+1}) \cdot
        \beta_{t+1,l}}
    \end{split}
  \end{equation}


\end{itemize}


\end{document}
