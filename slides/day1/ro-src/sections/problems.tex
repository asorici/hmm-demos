\begin{frame}
  \frametitle{Cele 3 Probleme ale MMA}
  \small{  
  3 probleme fundamentale \citep{rabiner1989tutorial}
  \begin{itemize}
	\item Particularizarea inferenței în probleme cu secvențe temporale pe cazul MMA
  	\item Structura restricționată a MMA permite implementări elegante ale tuturor algoritmilor de bază
  \end{itemize}
  }
  \pause

  \begin{block}{Problema Evaluării}
  	Dându-se un model și o secvență de observații, cum calculăm probabilitatea ca \alert{secvența observată} sa fi
  	fost produsă de model?
  \end{block}
  \pause
  
  \begin{block}{Problema Interpretării (cea mai bună explicație a observațiilor)}
  	Dându-se un model și o secvență de observații, cum alegem o \alert{secvență corespunzătoare de stări} care
  	\emph{dau sens} observațiilor?
  \end{block}
  \pause
  
  \begin{block}{Problema Estimării (Antrenării) Modelului}
	Dându-se mai multe secvențe de observații, cum putem ajusta \alert{parametrii modelului MMA} care explică cel 
	mai bine observațiile făcute?
  \end{block}

\end{frame}