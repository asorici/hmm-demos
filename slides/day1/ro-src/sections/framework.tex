\begin{frame}
  \frametitle{Mediul de lucru}
Pentru început
  \begin{enumerate}
  \item Deschideți \texttt{Matlab} / \texttt{Octave}
  \item Schimbați directorul de lucru cu \texttt{aria-hmm}
  \item Adăugați toate subdirectoarele în path:\\
    \mcode{addpath(genpath('.'))}
  \end{enumerate}
\end{frame}

\begin{frame}
  \frametitle{Platforma de lucru - fișierele stub}
  Fișierele \texttt{.stub}
  \begin{itemize}
  \item le veți folosi ca schelet de cod pentru sesiunile de implementare
  \item au sectiuni delimitate de \texttt{<label>-start} și \texttt{<label>-end}
    între care veți adăuga liniile de cod
  \item înlăturați sufixul \texttt{.stub} când rezolvați task-urile.
  \end{itemize}\vspace*{-1em}%
  \lstinputlisting{m-files/forward-backward.m}
\end{frame}

\begin{frame}
  \frametitle{Platforma de lucru - tester}
  \begin{itemize}
    \item Testerul se rulează folosind comanda \mcode{hmm_test}
      \lstinputlisting{m-files/output.m}
    \item Cu \mcode{list} afișați toate testele disponibile
    \item Pentru un anumit task numele testului coincide cu eticheta care delimitează secțiunea de completat.
  \end{itemize}
\end{frame}