\documentclass{article}

\usepackage[utf8x]{inputenc}
\usepackage[romanian]{babel}
\usepackage{algorithm}
\usepackage{algorithmic}
\usepackage{mytitlepage}
\usepackage{graphicx}

\title{Modele Markov Ascunse}
\titlesndline{De la Teorie la Aplicații}
\titledoc{Ghid pentru partea practică}
\author{Alexandru Sorici, Tudor Berariu}
\institute{Asociația Română pentru Inteligență Artificială}
\collaborator{Laboratorul AI-MAS}
\logosrc{graphics/aria-logo-small-white.png}

\begin{document}

\mytitlepage

\section{Mediul de lucru}
\label{sec:framework}

\subsection{Inițializarea mediului de lucru}
\label{sec:init}

\subsection{Fișierele \texttt{.stub}}
\label{sec:stubs}

\subsection{Testerul}
\label{sec:tester}


\section{Notații și Denumirile Variabilelor}
\label{sec:notations}

\section{Task-uri de implementare}
\label{sec:tasks}

\subsection{Algoritmul Forward-Backward}
\label{sec:fb}

\subsection{Algoritmul Viterbi}
\label{sec:viterbi}

\subsection{Algoritmul Baum-Welch}
\label{sec:baum-welch}

\subsection{Recunoașterea Simbolurilor}
\label{sec:symbol-recognition}


\section{Soluții}
\label{sec:solutions}

\subsection{Algoritmul Forward-Backward}
\label{sec:fb-sol}

\subsection{Algoritmul Viterbi}
\label{sec:viterbi-sol}

\subsection{Algoritmul Baum-Welch}
\label{sec:baum-welch-sol}

\subsection{Recunoașterea Simbolurilor}
\label{sec:symbol-recognition-sol}

\end{document}
